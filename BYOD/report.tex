\documentclass[a4paper, 12pt]{article}		% general format

%%%% Charset
\usepackage{cmap}							% make PDF files searchable and copyable
\usepackage[utf8x]{inputenc}				% accept different input encodings
\usepackage[T2A]{fontenc}					% Russian font
\usepackage[russian]{babel}					% multilingual support (T2A)

%%%% Graphics
\usepackage[dvipsnames]{xcolor}			% driver-independent color extensions
\usepackage{graphicx}						% enhanced support for graphics
\usepackage{wrapfig}						% produces figures which text can flow around

%%%% Math
\usepackage{amsmath}						% American Mathematical Society (AMS) math facilities
\usepackage{amsfonts}						% fonts from the AMS
\usepackage{amssymb}						% additional math symbols

%%%% Typography (don't forget about cm-super)
\usepackage{microtype}						% subliminal refinements towards typographical perfection
\linespread{1.3}							% line spacing
\usepackage[left=2.5cm, right=1.5cm, top=2.5cm, bottom=2.5cm]{geometry}
\setlength{\parindent}{0pt}					% we don't want any paragraph indentation
\usepackage{parskip}						% add distance between paragraphs

%%%% Tables
\usepackage{tabularx}						% Normal tables
\usepackage{multirow}						% for tabular
\usepackage{hhline}							% for tabular


%%%% Other
\usepackage{url}							% verbatim with URL-sensitive line breaks
\usepackage{fancyvrb}						% verbatim with box
\setcounter{secnumdepth}{5}					%

%------------------------------------------------------------------------------
\usepackage{listings}						% typeset source code listings

% Настройки отображения кода
\lstset{
	% Настройки отображения
	breaklines=true,						% Перенос длинных строк
	basicstyle=\ttfamily\footnotesize,		% Шрифт для отображения кода
	frame=tblr								% draw a frame at all sides of the code block
	tabsize=2,								% tab space width
	showstringspaces=false,					% don't mark spaces in strings
	% Настройка отображения номеров строк. Если не нужно, то удалите весь блок
	numbers=left,							% Слева отображаются номера строк
	stepnumber=1,							% Каждую строку нумеровать
	numbersep=5pt,							% Отступ от кода
	numberstyle=\small\color{black},		% Стиль написания номеров строк
}

% Для настройки заголовка кода
\usepackage{caption}
\renewcommand{\lstlistingname}{Листинг} % Переименование Listings в нужное именование структуры
%------------------------------------------------------------------------------
\begin{document}

\begin{titlepage}
\thispagestyle{empty}

\begin{center}
Санкт-Петербургский политехнический университет Петра Великого\\
Институт Информационных Технологий и Управления\\*
Кафедра компьютерных систем и программных технологий\\*
\hrulefill
\end{center}

\vspace{15em}

\begin{center}
\textsc{\textbf{Научно-исследовательская работа}}
\vspace{1em}

%Дисциплина: \textbf{Методы оптимизации}
\vspace{2em}

Тема: \textbf{Установка WSO2 Enterprise Mobility Manager}
\end{center}

\vspace{16em}

\begin{flushleft}
Выполнил студент гр. 53501/3 \hrulefill С.А. Мартынов \\
\vspace{1.5em}
Руководитель, к.т.н.,доц. \hrulefill И.В. Стручков\\
\end{flushleft}

\vspace{\fill}

\begin{center}
Санкт-Петербург \\
2015
\end{center}

\end{titlepage}
\setcounter{page}{2}

\tableofcontents{}
\newpage
%------------------------------------------------------------------------------

\section{Введение}

Число пользователей мобильных устройств в последние годы продолжает расти. На смену ноутбукам пришли смартфоны и планшеты. Эти устройства сопровождают своих владельцев как дома так и на рабочем месте, где не редко используются для решения служебных задач. Популярность широкого применения персональных мобильных устройств в рабочих целях достаточно легко объяснить: тот уровень удобства, к которому пользователи привыкают в повседневной жизни, формирует их требования к приложениям и устройствам, которые им предстоит использовать на работе. Работодатели часто стремятся повысить мобильность своих сотрудников, выдавая им корпоративные мобильные устройства. Но, получив определённый "пользовательский опыт" со своим персональным устройством, сотрудник начинает испытывать раздражение, столкнувшись с ограничениями, налогаемыми другим типом устройства или платформы. Компания, в свою очередь, не может обеспечить сотрудника желаемым устройством, т.к. закупка техники должна подвергаться опредёлённым правилам (к примеру, продукция одного вендора, предоставляющего особые условия оплаты и обслуживания). В итоге у работодателя остаётся два варианта: либо навязывать корпоративные решения сотрудникам, либо позволить им работать со своими устройствами. Попыткой решения данного вызова стало формирование определённого класса программного обеспечения, позволявшего сотрудникам использовать любые устройства для комфортной работы, а работодателю гарантировать требуемый уровень безопасности корпоративных данных и инфраструктуры. Это явление получило название BYOD (Bring Your Own Device, принеси свое собственное устройство) и изначально развивалось в США. Впоследствии, BYOD превратилось в глобальную тенденцию, распространившуюся по всему миру.

Подразделении Internet Business Solutions Group (IBSG) компании Cisco в 2012-м году провело масштабные исследования распространения BYOD в бизнес-среде, в котором участвовали крупные и средние предприятия из восьми стран трех регионов (таким образом, было опрошено почти 4900 ИТ-руководителей из 18 отраслевых секторов). По результатам исследования \cite{PCWeek30} было выявлено, что мобильными устройствами в своей работе пользуется 60\% персонала. Из этих устройств 42\% смартфонов и 38\% ноутбуков принадлежат самим сотрудникам, а не работодателю, при этом в дальнейшем ожидается увеличение этих значений. Вместе с ростом количества устройств ожидается и рост числа неразрешенных приложений, которые сотрудники могут получить через магазины приложений Apple и Google, и которые могут заменить собой защищённые корпоративные сервисы.

Руководители корпоративных ИТ-отделов в большинстве случаев (90\%) позитивно настроены относительно BYOD и признают рост производительности персонала. Вместе с тем, это может стать и причиной дополнительных трат для компании. К примеру, использование персоналом различных медиа сервисов, включающих мультимедийные компоненты, может создать дополнительную нагрузку на корпоративную сеть и потребуется решать проблемы возникновения "узких" мест. Другой статьёй расходов могут стать риски, связанные с безопасностью сети, и защитой корпоративных данных. Рост количества устройств усложняет процесс их контроля, становится сложнее определить, какие устройства являются легальными, а какие нет.

Озабоченность вопросами безопасности подтвердило исследование, проведенное в конце 2014 года агентством MSI Research по заказу Intel Security среди 2500 сотрудников 18–65 лет в 12 странах. Из отчёта \cite{PCWeekPR} следует, что 78\% респондентов используют личные устройства в рабочих целях, примерно то же число сотрудников (79\%) поступают наоборот -- используют рабочие устройства для своих личных целей. Многие сотрудники признали, что часто работаю из дома (40\%), а некоторые (35\%) используют публичные WiFi сети для подключения к корпоративным ресурсам.

Популярность BYOD растет: по данным Gartner \cite{Gartner}, к 2016 году 38\% опрошенных компаний предполагают прекратить предоставлять устройства (в т.ч. настольные ПК) персоналу для работы, а к 2017 году половина работодателей будет приветствовать готовность работать на личных устройствах, что даёт больше свободы сотрудникам и отчасти оптимизирует затраты на ИТ.

Но это значительно повышает уровень требований к устройствам сотрудников! По данным Intel Security \cite{PCWeekPR}, 61\% сотрудников исполняют задачи, связанные с высокой степенью конфиденциальности или приватности данных. При этом 65\% респондентов уверены, что защита персональных данных на рабочем устройстве является обязанностью ИТ-отдела, а 77\% убеждены в том, что работодатель предпринимает все необходимые шаги для защиты критически важных данных. Сотрудники заняты своими непосредственными обязанностями и не имеют времени беспокоиться о защите информации, но часто их компании-работодатели подвергаются угрозе именно из-за недостаточной подготовленности и информированности рядовых сотрудников.

Таким образом, на ряду с очевидными экономическими выгодами, BYOD несёт в себе новые угрозы безопасности, которые IT-отделы должны научиться эффективно отражать.

%------------------------------------------------
\section{Список литературы}

\begin{thebibliography}{00}

\bibitem{PCWeek30} Максимов А. BYOD -- глобальная растущая тенденция. // PC Week/RE. 2012. №30.
% http://www.pcweek.ru/infrastructure/article/detail.php?ID=143991

\bibitem{PCWeekPR}Результаты опроса Intel Security о тенденциях BYOD. Пресс-релиз. // PC Week/RE. 2015.
% http://www.pcweek.ru/security/news-company/detail.php?ID=170258

\bibitem{Gartner}Bring Your Own Device: The Facts and the Future. URL: \url{http://www.gartner.com/resId=2422315}
% http://www.gartner.com/newsroom/id/2466615

\bibitem{miller2012} Miller K.W., Voas J., Hurlburt G.F. BYOD: Security and privacy considerations. // It Professional. 2012. №1520-9202/12. C. 53 - 55.

\bibitem{thomson2012}Thomson G. BYOD: enabling the chaos. // Network Security. 2012. №2. С. 5 - 8.

\bibitem{eslahi2014} Eslahi M., Naseri M.V., Hashi H. BYOD: Current state and security challenges. 2014.

\bibitem{garba2015} Garba A.B., Armarego J., Murray D. Review of the information security and privacy challenges in Bring Your Own Device (BYOD) environments. // Journal of Information. 2015.


\end{thebibliography}

%------------------------------------------------------------------------------

\end{document}